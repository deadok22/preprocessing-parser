\documentclass{beamer}
\usepackage[utf8]{inputenc}
\usepackage[russian]{babel}

\usecolortheme{orchid}

\setbeamertemplate{footline}[frame number]

\title{Синтаксический анализ исходного кода, содержащего инструкции препроцессора}
\author{Савенко Сергей Андреевич}
\institute{{\tiny научный руководитель} \\ \vspace{.10cm} Игнатов Сергей Сергеевич}
\date{\scriptsize Академический университет\\ \vspace{.10cm} 2014 г.}



\begin{document}

\frame{\titlepage}

\frame{
  \frametitle{Текстовый препроцессор}
  
  \begin{block}{Типичная функциональность}
    \begin{itemize}
      \item Включение файлов
      \item Макроопределения
      \item Блоки условной компиляции
    \end{itemize}
  \end{block}

  \begin{block}{Сложности при обработке непрепроцессированного кода}
    \begin{itemize}
      \item 2 языка
      \item Зависимость кода от конфигурации препроцессора
    \end{itemize}
  \end{block}
}

\frame{
  \frametitle{Пример непрепроцессированного кода}
  TODO write some C or Erlang code here
}

\frame{
  \frametitle{Обработка непрепроцессированного кода}

  \begin{block}{Существующие подходы}
    \begin{enumerate}
      \item Обработка конкретных конфигураций препроцессора
      \item Эвристическая обработка
      \item Обработка всех возможных конфигураций препроцессора
    \end{enumerate}
  \end{block}
}

\frame{
  \frametitle{Цели и задачи}

  \begin{block}{Цель работы}
    Создать инструмент, позволяющий производить синтаксический анализ исходного кода, содержащего инструкции препроцессора, используя грамматику языка программирования и интерпретатор инструкций препроцессора
  \end{block}

  \begin{block}{Задачи работы}
    \begin{enumerate}
      \item Изучить существующие решения
      \item Разработать и реализовать алгоритм синтаксического анализа непрепроцессированного кода
      \item Проверить разработанную библиотеку, сравнив её с аналогами
    \end{enumerate}
  \end{block}
}

\frame{
  \frametitle{Существующие решения}

  \begin{block}{SuperC}
    \begin{itemize}
      \item LR-анализатор
      \item Парсер-генератор Bison
    \end{itemize}
  \end{block}

  \begin{block}{TypeChef}
    \begin{itemize}
      \item LL-анализатор
      \item Библиотека парсер-комбинаторов
    \end{itemize}
  \end{block}
}

\frame{
  \frametitle{Подход к разбору непрепроцессированного кода}

  Синтаксический анализ проводится в 2 этапа:
  
  \begin{block}{Частичное препроцессирование}
    \begin{itemize}
      \item Интерпретация инструкций препроцессора
      \item Сохранение информации об условных ветвлениях
      \item Результат: направленный ориентированный граф последовательностей лексем для всех ветвей условной компиляции
    \end{itemize}
  \end{block}
  
  \begin{block}{Синтаксический анализ графа лексем}
    \begin{itemize}
      \item Обработка ветвей условной компиляции
      \item Результат: синтаксическое дерево, содержащее узлы условного ветвления
    \end{itemize}
  \end{block}
}

\frame{
  \frametitle{Модифицированный алгоритм Earley}
 
  Строит таблицу из колонок, содержащих множества элементов Earley 

  \begin{block}{Элемент Earley}
    Кортеж, состоящий из:
    \begin{itemize}
      \item Правила грамматики и индекса в нем
      \item Номера колонки, в которой началось сопоставление правила этого элемента
      \item Множества элементов-предшественников
      \item Условия наличия 
    \end{itemize}
  \end{block}
}


\frame{
  \frametitle{Сравнение с TypeChef}
 
  Сравнение проводилось на подмножестве языка Erlang 

  \begin{block}{Отличия разработанного решения}
    \begin{itemize}
      \item Отсутствие необходимости модификации грамматики языка
      \item Отсутствие необходимости описания ветвлений и слияний
      \item Лаконичность синтаксиса описания языка: $\sim$100 строк на Java против $\sim$200 строк на Scala
      \item Сравнимая скорость работы: медианное время работы на тестовых данных в $\sim$2 раза меньше
    \end{itemize}
  \end{block}
}


\frame{
  \frametitle{Результаты}
  
  Разработан инструмент для создания синтаксических анализаторов исходного кода, содержащего инструкции препроцессора

  \begin{block}{Особенности решения}
    \begin{itemize}
      \item Описание грамматики языка в БНФ
      \item Автоматическое ветвление и слияние ветвей разбора
      \item Поддержка широкого класса языков (все КС-языки)
      \item Сравнимая с аналогами производительность
    \end{itemize}
  \end{block} 
}

\end{document}
