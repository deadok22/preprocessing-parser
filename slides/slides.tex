\documentclass{beamer}
\usepackage[utf8]{inputenc}
\usepackage[russian]{babel}

\usecolortheme{orchid}

\setbeamertemplate{footline}[frame number]

\title{Синтаксический анализ исходного кода, содержащего инструкции препроцессора}
\author{Савенко Сергей Андреевич}
\institute{{\tiny научный руководитель} \\ \vspace{.10cm} Игнатов Сергей Сергеевич}
\date{\scriptsize Академический университет\\ \vspace{.10cm} 2014 г.}



\begin{document}

\frame{\titlepage}

\frame{
  \frametitle{Синтаксический анализ}

  \begin{block}{Способы реализации парсера}
    \begin{itemize}
      \item Вручную
      \item Парсер-генераторы (ANTLR, Bison, JavaCC и пр.)
      \item Парсер-комбинаторы
    \end{itemize}
  \end{block}
}

\frame{
  \frametitle{Текстовый препроцессор}
  
  \begin{block}{Типичная функциональность}
    \begin{itemize}
      \item Включение файлов
      \item Блоки условной компиляции
      \item Макроопределения
    \end{itemize}
  \end{block}

  \begin{block}{Сложности при синтаксическом анализе}
    \begin{itemize}
      \item 2 языка
      \item Множество конфигураций препроцессора
    \end{itemize}
  \end{block}
}

\frame{
  \frametitle{Обработка непрепроцессированного кода}

  \begin{block}{Существующие подходы}
    \begin{enumerate}
      \item Обработка конкретных конфигураций препроцессора
      \item Эвристическая обработка
      \item Обработка всех возможных конфигураций
    \end{enumerate}
  \end{block}
}

\frame{
  \frametitle{Цели и задачи}

  \begin{block}{Цель работы}
    Создать инструмент, позволяющий производить синтаксический анализ исходного кода, содержащего инструкции препроцессора, используя грамматику языка программирования и интерпретатор инструкций препроцессора
  \end{block}

  \begin{block}{Задачи работы}
    \begin{enumerate}
      \item Изучить существующие решения
      \item Разработать и реализовать алгоритм синтаксического анализа непрепроцессированного кода
      \item Проверить разработанную библиотеку, сравнив её с аналогами
    \end{enumerate}
  \end{block}
}

\frame{
  \frametitle{Синтаксический анализ непрепроцессированного кода}

  \begin{block}{Частичное препроцессирование}
    Используется подход применяемый в аналогах:
    \begin{itemize}
      \item Таблица макроопределений
      \item Задача достигающих определений
      \item Граф из последовательностей лексем для ветвей условной компиляции
    \end{itemize} 
  \end{block}

  \begin{block}{Синтаксический анализ графа лексем}
    \begin{itemize}
      \item Модифицированный алгоритм Earley
      \item Обработка условий наличия лексем
      \item Граф из узлов, соответствующих символам грамматики, и узлов ветвления
    \end{itemize}
  \end{block}
}

\frame{
  \frametitle{Особенности реализации}

  \begin{itemize}
    \item Описание грамматики языка в БНФ
    \item Поддержка широкого класса языков (все КС-языки)
    \item Автоматическое ветвление и слияние ветвей разбора
  \end{itemize}
}

\frame{
  \frametitle{Результаты}
  
  Для оценки результатов было описано подмножество языка Erlang для разработанной библиотеки и для аналогичного решения TypeChef
 
  \begin{block}{Свойства разработанного решения} 
    \begin{itemize}
      \item Простота в использовании 
        \begin{itemize}
          \item Отсутствие необходимости модифицировать грамматику (поддержка более широкого класса языков)
          \item Отутствие необходимости описания ветвлений и слияний в грамматике
          \item Лаконичность: ~100 строк на Java против ~200 строк на Scala
        \end{itemize}
      \item Сравнимая скорость работы
    \end{itemize}
  \end{block}
}

\end{document}
