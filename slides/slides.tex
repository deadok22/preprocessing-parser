\documentclass{beamer}
\usepackage[utf8]{inputenc}
\usepackage[russian]{babel}

\usecolortheme{orchid}

\setbeamertemplate{footline}[frame number]

\title{Синтаксический анализ исходного кода, содержащего инструкции препроцессора}
\author{Савенко Сергей Андреевич}
\institute{{\tiny научный руководитель} \\ \vspace{.10cm} Игнатов Сергей Сергеевич}
\date{\scriptsize Академический университет\\ \vspace{.10cm} 2014 г.}



\begin{document}

\frame{\titlepage}

\frame{
  \frametitle{Синтаксический анализ}

  \begin{block}{Способы реализации парсера}
    \begin{itemize}
      \item Написание вручную
      \item Парсер-генераторы (ANTLR, Bison, JavaCC и пр.)
      \item Парсер-комбинаторы
    \end{itemize}
  \end{block}
}

\frame{
  \frametitle{Текстовый препроцессор}
  
  \begin{block}{Типичная функциональность}
    \begin{itemize}
      \item Включение файлов
      \item Блоки условной компиляции
      \item Макроопределения
    \end{itemize}
  \end{block}

  \begin{block}{Сложности при синтаксическом анализе}
    \begin{itemize}
      \item 2 языка
      \item Множество конфигураций препроцессора
    \end{itemize}
  \end{block}
}

\frame{
  \frametitle{Обработка непрепроцессированного кода}

  \begin{block}{Существующие подходы}
    \begin{enumerate}
      \item Обработка конкретных конфигураций препроцессора
      \item Эвристическая обработка
      \item Обработка всех возможных конфигураций
    \end{enumerate}
  \end{block}
}

\frame{
  \frametitle{Цели и задачи}

  \begin{block}{Цель работы}
    Создать инструмент, позволяющий производить синтаксический анализ исходного кода, содержащего инструкции препроцессора, используя грамматику языка программирования и интерпретатор инструкций препроцессора
  \end{block}

  \begin{block}{Задачи работы}
    \begin{enumerate}
      \item Изучить существующие решения
      \item Разработать и реализовать алгоритм синтаксического анализа непрепроцессированного кода
      \item Проверить разработанную библиотеку, сравнив её с аналогами
    \end{enumerate}
  \end{block}
}

\end{document}
