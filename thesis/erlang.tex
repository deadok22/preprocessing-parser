\clearpage

\section{Синтаксический анализ кода на языке Erlang}
\label{sec:erlang}

Для проверки работоспособности разработанной библиотеки для создания синтаксических анализаторов исходного кода, содержащего инструкции препроцессора, было предложено реализовать синтаксический анализатор для подмножества языка Erlang. 

Синтаксический анализатор для подмножества языка Erlang, созданный с помощью разработанной библиотеки, является прототипом синтаксического анализатора, который будет использован в проекте intellij-erlang\footnote{\url{https://github.com/ignatov/intellij-erlang}} --- интегрированной среде разработки для языка Erlang на основе IntelliJ IDEA\footnote{\url{http://www.jetbrains.com/idea/}}.

На данный момент в intellij-erlang используется синтаксический анализатор, сгенерированный при помощи Grammar-Kit\footnote{\url{https://github.com/JetBrains/Grammar-Kit}}. Инструкции препроцессора в этом анализаторе включены в грамматику языка с тем, чтобы обеспечить возможность разбора наиболее часто встречающихся паттернов использования препроцессора. Однако, такой подход не позволяет производить анализ кода, содержащего некоторые варианты использования препроцессора, а также производить поиск ошибок, связанных с подстановкой макроопределений и условной компиляцией. 

В этом разделе описываются особенности реализации синтаксического анализатора исходного кода на подмножестве языка Erlang, содержащем инструкции препроцессора, c использованием библиотеки, описанной в разделе \ref{sec:parserlibimpl}.

\subsection{Препроцессор}

Здесь будет приведено краткое описание функциональности препроцессора языка Erlang (за полным описанием следует обратиться к \cite{erlangpreprocessor}) а также особенностей её реализации с использованием разработанной библиотеки.

\subsubsection{Включение файлов}

Препроцессор языка Erlang позволяет производить подстановку текстовых файлов, имена которых разрешаются при помощи двух разных способов:

\begin{itemize}
\item Имя файла, включенного при помощи аттрибута include, разрешается относительно каталога, из которого этот файл был подключен;
\item Имя файла, включенного при помощи аттрибута include\_lib, разрешается относительно набора каталогов, содержащих зависимости данного проекта.
\end{itemize}

Оба способа легко реализуются в рамках интерфейсов, предоставляемых библиотекой. Непосредственно подстановка текста реализована в самой библиотеке и не потребовала каких-либо усилий.

\subsubsection{Условная компиляция}

Для реализации функциональности, связанной с условной компиляцией, такой как условное ветвление и подстановки макроопределений, библиотека требует реализации условий наличия.

Условия наличия в случае Erlang представляют собой формулы логики высказываний, где переменным соответствуют утверждения об объявленности макроопределений. Формулы в реализации препроцессора для Erlang хранятся в дизъюнктивной нормальной форме.

\subsubsection{Макроопределения}

Абстракции, предоставляемые библиотекой, позволили без труда реализовать объявления и разобъявления макроопределений, а также парсеры их вызовов, что позволило получить работающую подстановку макроопределений.

\subsection{Лексер}

В качестве лексера языка Erlang был использован несколько модифицированный лексер из проекта intellij-erlang, генерируемый при помощи сканнер-генератора JFlex\footnote{\url{http://jflex.de/}}.

Для использования сгенерированного лексера понадобилось только реализовать адаптер (см. \cite{gangoffouradapter}) для того, чтобы сгенерированный лексер реализовывал интерфейс, предоставляемый библиотекой.

\subsection{Парсер}

Реализация парсера средствами разработанной библиотеки состояла в переписывании правил грамматики языка Erlang\footnote{\url{https://github.com/erlang/otp/blob/master/lib/stdlib/src/erl\_parse.yrl}}, описанной в BNF, на предметно-ориентированном языке, пример использования которого приведен в листинге \ref{plus-grammar}.

Никаких действий по адаптации грамматики для использования с разработанной библиотекой не потребовалось, --- переписанные правила имеют тот же вид, что и правила исходной грамматики.

