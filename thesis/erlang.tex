\clearpage

\section{Синтаксический анализ кода на языке Erlang}

Для проверки работоспособности библиотеки для разработки синтаксических анализаторов исходного кода, содержащего инструкции препроцессора, с её помощью был разработан синтаксический анализатор, позволяющий разбирать некоторое подмножество языка Erlang.

Результаты работы компонент библиотеки разработки синтаксических анализаторов, а также компонент специально реализованных для анализа исходного кода на языке Erlang проверялись с помощью набора тестов для тестового фреймворка JUnit\footnote{\url{http://junit.org/}}.

\subsection{Препроцессор}

Здесь будет приведено краткое описание функциональности препроцессора языка Erlang (за полным описанием следует обратиться к \cite{erlangpreprocessor}) а также особенностей её реализации с использованием разработанной библиотеки.

\subsubsection{Включение файлов}

Препроцессор языка Erlang позволяет производить подстановку текстовых файлов, имена которых разрешаются при помощи двух разных способов:

\begin{itemize}
\item Имя файла, включенного при помощи аттрибута include, разрешается относительно каталога, из которого этот файл был подключен;
\item Имя файла, включенного при помощи аттрибута include\_lib, разрешается относительно набора каталогов, содержащих зависимости данного проекта.
\end{itemize}

Оба способа легко реализуются в рамках интерфейсов, предоставляемых библиотекой. Непосредственно подстановка текста реализована в самой библиотеке и не потребовала каких-либо усилий.

\subsubsection{Условная компиляция}

Для реализации функциональности, связанной с условной компиляцией, такой как условное ветвление и подстановки макроопределений, библиотека требует реализации условий наличия.

Условия наличия в случае Erlang представляют собой пропозициональные формулы, где пропозициональным переменным соответствуют утверждения об объявленности макроопределения. Формулы в реализации препроцессора для Erlang хранятся в дизъюнктивной нормальной форме.

\subsubsection{Макроопределения}

Абстракции, предоставляемые библиотекой, позволили без труда реализовать объявления и разобъявления макроопределений, а также парсеры их вызовов, что позволило получить работающую подстановку макроопределений.

Однако, библиотека требует доработки, так как, на данный момент, она не позволяет использовать параметры макроопределений, а также использовать предъобъявленные макроопределения.

\subsection{Лексер}

В качестве лексера языка Erlang был использован несколько модифицированный лексер из проекта intellij-erlang\footnote{\url{https://github.com/ignatov/intellij-erlang}}, генерируемый при помощи сканнер-генератора JFlex\footnote{\url{http://jflex.de/}}.

Адаптация лексера для использования библиотекой не потребовала больших усилий --- понадобилось лишь реализовать соответствующие интерфейсы.

\subsection{Парсер}

Реализация парсера состоит в использовании средств, предоставляемых библиотекой, для описания грамматики языка.

Была использована часть грамматики языка Erlang\footnote{\url{https://github.com/erlang/otp/blob/master/lib/stdlib/src/erl\_parse.yrl}}. В целом, описание грамматики с помощью средств, предоставляемых библиотекой, не составляет труда --- достаточно просто переписать грамматику "как есть".

В процессе описания грамматики был обнаружен следующий недостаток: библиотека не позволяет описывать правила, содержащие пустую строку. Однако, это устранимый недостаток --- известны способы реализации этой функциональности для парсеров Earley\cite{emptyrules}.

