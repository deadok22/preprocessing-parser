\clearpage
\section{Введение}

Для создания многих средств для работы с исходным кодом - таких, как браузеры исходного кода, инструменты автоматического поиска ошибок, среды разработки, компиляторы требуется разработать синтаксический анализатор - инструмент, позволяющий сопоставить входной текст с грамматикой языка с тем, чтобы получить дерево синтаксического разбора\cite{aho}.

Задача разработки синтаксических анализаторов (парсеров) на сегодняшний день решается одним из следующих способов:
\begin{itemize}
	\item написание парсера вручную;
	\item использование средств автоматической генерации синтаксическх анализаторов таких, например, как ANTLR\footnote{\url{http://www.antlr.org/}}, Bison\footnote{\url{http://www.gnu.org/software/bison/}}, JavaCC\footnote{\url{https://javacc.java.net/}};
\end{itemize}
Использование парсер-генераторов позволяет существенно снизить трудозатраты при разработке инструментов обработки естественных и искуственных языков.

Часто при разработке программных продуктов в исходный код на языке программирования добавляются инструкции препроцессора, позволяющие повысить выразительность используемого языка программирования посредством текстовых подстановок из других файлов, осуществения макроподстановок, использования условной компиляции. Непрепроцессированный код также может зависеть от переменных окружения, переданных препроцессору, - тогда в одном файле с исходным кодом может быть описано сразу несколько различных программ. Это используется для создания целых семейств программных продуктов с общей кодовой базой\cite{flightsoftwareproductline}, а также для создания высоко конфигурируемых программных систем. Использование текстовых препроцессоров при разработке программного обеспечения не является экзотикой - в некоторых языках, например C, C++, C\#, препроцессирование исходного кода является одним из этапов компиляции.

Применение препроцессора приводит и к появлению ошибок, проявляющихся только при компиляции в некоторых конкретных конфигурациях --- это могут быть синтаксические ошибки, ошибки типизации, ошибки при разрешении зависимостей и многие другие типы ошибок. Это приводит к необходимости обработки инструкций препроцессора в инструментах анализа исходного кода.

Для разработки таких инструментов требуются синтаксические анализаторы, способные обрабатывать "смесь" из инструкций препроцессора и конструкций целевого языка программирования. Таким парсерам необходимо обрабатывать случаи "прерывания" грамматических конструкций языка инструкциями препроцессора, производить макроподстановки, обрабатывать варианты программы, которые могут быть получены при различных конфигурациях препроцессора.

Синтаксические анализаторы для языка С, обладающие описанной функциональностью, описаны в работах \cite{superc} и \cite{typechef}. В данной работе представлен инструмент для упрощения разработки таких синтаксических анализаторов. 
