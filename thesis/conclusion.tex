\clearpage

\section{Заключение}

Настоящая работа продемонстрировала возможность создания инструмента, позволяющего производить синтаксический анализ исходного кода, содержащего инструкции препроцессора, используя синтаксический анализатор исходного кода языка препроцессора, средства интерпретации его инструкций и грамматику языка программирования.

В работе предложена модификация алгоритма синтаксического анализа Earley, использование которой позволяет восстанавливать синтаксические деревья, соответствующие всем возможным конфигурациям препроцессора.

Предложенный алгоритм был реализован в библиотеке позволяющей создавать синтаксические анализаторы исходного кода, содержащего инструкции препроцессора. Библиотека была применена для синтаксического анализа исходного кода на подмножестве языка Erlang.

Было произведено сравнение разработанной библиотеки с аналогичным решением TypeChef --- разработанное решение позволяет с меньшими трудозатратами реализовать синтаксический анализатор исходного кода, содержащего инструкции препроцессора, обладающий сравнимыми характеристиками производительности.

Использование подхода предложенного в этой работе позволяет существенно сократить время на разработку синтаксического анализатора исходного кода, содержащего инструкции препроцессора.

Полученные результаты могут быть использованы для разработки средств автоматической генерации синтаксических анализаторов для языков программирования, используемых с препроцессором.
